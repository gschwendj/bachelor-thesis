%% Custom commands
%% ===============

%% Special characters for number sets, e.g. real or complex numbers.
\newcommand{\C}{\mathbb{C}}
\newcommand{\K}{\mathbb{K}}
\newcommand{\N}{\mathbb{N}}
\newcommand{\Q}{\mathbb{Q}}
\newcommand{\R}{\mathbb{R}}
\newcommand{\Z}{\mathbb{Z}}
\newcommand{\X}{\mathbb{X}}
\newcommand{\F}{\mathcal{F}}

%% Fixed/scaling delimiter examples (see mathtools documentation)
\DeclarePairedDelimiter\abs{\lvert}{\rvert}
\DeclarePairedDelimiter\norm{\lVert}{\rVert}

%% Use the alternative epsilon per default and define the old one as \oldepsilon
\let\oldepsilon\epsilon
\renewcommand{\epsilon}{\ensuremath\varepsilon}

%% Also set the alternate phi as default.
\let\oldphi\phi
\renewcommand{\phi}{\ensuremath{\varphi}}
\newcommand{\mmod}[1]{\quad\left(\text{mod} #1\right)}
\newcommand{\note}[1]{}
\newcommand{\pos}[1]{\ensuremath{P_{#1}(x_{#1},y_{#1},z_{#1})}}
\newcolumntype{_}{>{\global\let\currentrowstyle\relax}}
\newcolumntype{^}{>{\currentrowstyle}}
\newcommand{\rowstyle}[1]{\gdef\currentrowstyle{#1}%
  #1\ignorespaces
}
\newcolumntype{Y}{>{\centering\arraybackslash}X}

\newcommand{\td}{{\color{red}TODO}}
\newcommand{\cref}[1]{\textit{\ref{#1} \titleref{#1}}}
%\newcommand{\abs}[1]{\left\lvert #1\right\rvert}

\DeclareMathOperator{\rang}{rang}
\DeclareMathOperator{\Kern}{kern}
%\DeclareMathOperator{\dim}{dim}



%\DeclarePairedDelimiterX{\norm}[1]{\lVert}{\rVert}{#1}