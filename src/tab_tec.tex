\begin{table}%[!t]
	\renewcommand{\arraystretch}{1.3}
%	\centering
	\begin{tabularx}{\textwidth}{|_p{0.12\textwidth}|^X|^X|}
		\hline\rowstyle{\bfseries}
		Technik & Vorteile & Nachteile \\\hline
	RSSI & Einfach zu implementieren, kosteneffizient, kann mit verschiedenen Technologien verwendet werden & Anfällig gegenüber Mulitpath Fading und Umgebungsrauschen, schlechte Genauigkeit, benötigt evtl. Fingerprinting\\\hline
	CSI & Besseres Verhalten bei Multipathing und Indoorrauschen als RSSI & Schlechte Verfügbarkeit \\\hline
	AoA	& Hohe Lokalisierungsgenauigkeit, benötigt kein Fingerprinting & Benötigt evtl. direktionale Antennen und komplexe Hardware, benötigt komplexe Algorithmen und die Performance verschlechtert sich auf die Distanz \\\hline
	ToF & Hohe Lokalisierungsgenauigkeit, benötigt kein Fingerprinting & Benötigt Zeitsynchronisation zwischen Empfängern und Sendern, evtl Zeitstempel und mehrere Antennen bei Sendern und Empfängern. Line of Sight zwingend für gute Performance \\\hline
	TDoA & Benötigt kein Fingerprinting und keine Zeitsynchronisation zwischen Ziel und Reference Nodes & Benötigt Clocksynchronisation zwischen den Reference Nodes, evtl Zeitstempel, grössere Bandbreite benötigt\\\hline
	RToF & Hohe Lokalisierungsgenauigkeit, benötigt kein Fingerprinting & Benötigt Clocksynchronisierung, Processing Delay kann die Performance auf kurze Distanzen beeinflussen\\\hline
	Finger-\newline printing & Einfach in der Anwendung & Neuer Fingerprint benötigt, sobald sich das Umfeld nur geringfügig verändert\\\hline
	\end{tabularx}
	\caption{Vor- und Nachteile unterschiedlicher Lokalisierungstechniken \cite{faheemzafari}}
	\label{tab:evaltec}
\end{table}