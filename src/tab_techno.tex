%\begin{table}%[t]
\begin{sidewaystable}
	\renewcommand{\arraystretch}{1.3}
%	\centering
	\begin{tabularx}{\textwidth}{|_p{0.115\textwidth}|^p{0.115\textwidth}|^p{0.08\textwidth}|^p{0.09\textwidth}|^p{0.22\textwidth}|^X|}
		\hline\rowstyle{\bfseries}
		Technologie & 
		Maximale\newline Reichweite &
		Daten-\newline durchsatz &
		Energie-\newline verbrauch & 
		Vorteile & 
		Nachteile\\\hline
		%
		IEEE 802.11 n\newline IEEE 802.11 ac\newline IEEE 802.11 ad & 
		250 m outdoor\newline 35 m indoor\newline 10-30 m & 
		600 Mbps\newline 1.3 Gbps\newline 4.6 Mbps & 
		Mittel & 
		Hohe Verfügbarkeit,\newline hohe Genauigkeit,\newline keine komplexe Hardware & 
		Anfällig gegenüber Rauschen,\newline benötigt komplexe Algorithmen \\\hline
		%
		IEEE 802.11 ah & 
		1 km & 
		100 Kbps & 
		Sehr gering & 
		Geringer Energieverbrauch,\newline grosse Reichweite & 
		Kaum Daten über Lokalisierung und Indoor Performance\\\hline
		%
		Bluetooth & 
		100 m & 
		24 Mbps & 
		Gering & 
		Geringer Energieverbrauch,\newline grosse Reichweite & 
		Genauigkeit eher gering,\newline anfällig gegenüber Rauschen\\\hline
		%
		UWB & 
		10-20 m & 
		460 Mbps & 
		Mittel & 
		Hohe Genauigkeit,\newline keine Interferenzen & 
		Kurze Reichweite,\newline hohe Kosten \\\hline
		%
		RFID & 
		200 m & 
		1.67 Gbps & 
		Gering & 
		Geringer Energieverbrauch,\newline grosse Reichweite & 
		Genauigkeit eher gering \\\hline
		%%
%		LoRA & 
%		15 km & 
%		%37.5 kbps & 
%		Sehr gering & 
%		Geringer Energieverbrauch, grosse Reichweite 
%		& \td\\\hline
%	SigFox & 50km & 100bps & Sehr gering & Geringer Energieverbrauch, grosse Reichweite & \td\\\hline
%	Weightless & 2-5 km & 10 kbps -\newline 10 Mbps & Sehr gering & Geringer Energieverbrauch, grosse Reichweite  & \td\\\hline
	\end{tabularx}
	\caption{Übersicht unterschiedlicher Wireless Technologien zur Lokalisierung \cite{faheemzafari}}
	\label{tab:evaltechno}
\end{sidewaystable}
%\end{table}
